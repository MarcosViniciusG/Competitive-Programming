\documentclass{article}
\usepackage{listings}
\usepackage{xcolor}
\usepackage{amsmath}

\lstdefinestyle{cppStyle}{
  language=C++,
  basicstyle=\ttfamily\small,
  numbers=left,
  numberstyle=\tiny\color{gray},
  stepnumber=1,
  numbersep=10pt,
  backgroundcolor=\color{white},
  showspaces=false,
  showstringspaces=false,
  showtabs=false,
  frame=single,
  tabsize=4,
  breaklines=true,
  breakatwhitespace=false,
  keywordstyle=\color{blue},
  commentstyle=\color{green!50!black},
  stringstyle=\color{red},
}

\begin{document}
\tableofcontents 

\section{Template}
\lstinputlisting[style=cppStyle]{Algorithms/template.hpp}

\section{Search}
\subsection{Ternary Search}
$O(\log{n})$

Function f(x) is unimodal on an interval [l, r]. Unimodal means: the function strictly increases first, reaches a maximum, and then strictly decreases OR the function strictly decreases first, reaches a minimum and then strictly decreases
\lstinputlisting[style=cppStyle]{Algorithms/Search/Ternary_Search/main.cpp}
\section{Algebra}
\subsection{All divisors}
$O(\sqrt{n})$
\lstinputlisting[style=cppStyle]{Algorithms/Algebra/divisors/main.cpp}

\subsection{Primality test}
$O(\sqrt{n})$
\lstinputlisting[style=cppStyle]{Algorithms/Algebra/primality_test/main.cpp}

\subsection{Binary exponentiation}
$O(\log n)$
\lstinputlisting[style=cppStyle]{Algorithms/Algebra/binary_exponentiation/main.cpp}

\subsection{Greatest common divisor}
$O(\log \min (a, b))$
\lstinputlisting[style=cppStyle]{Algorithms/Algebra/gcd/main.cpp}
\subsubsection{Least common multiple}
\lstinputlisting[style=cppStyle]{Algorithms/Algebra/gcd/lcm/main.cpp}

\section{Graphs}
\subsection{DFS}
$O(n+m)$
\lstinputlisting[style=cppStyle]{Algorithms/Graphs/dfs/main.cpp}

\subsection{BFS}
$O(n+m)$
\lstinputlisting[style=cppStyle]{Algorithms/Graphs/bfs/main.cpp}

\subsubsection{Shortest path on unweighted graph}
$O(n+m)$
\lstinputlisting[style=cppStyle]{Algorithms/Graphs/bfs/shortest_path_unweighted/main.cpp}

\subsection{Flood Fill}
$O(n+m)$
\lstinputlisting[style=cppStyle]{Algorithms/Graphs/flood_fill/main.cpp};

\subsection{Topological Sort (Directed Acyclic Graph)}
\subsubsection{DFS Variation}
$O(n+m)$
\lstinputlisting[style=cppStyle]{Algorithms/Graphs/topological_sort/dfs/main.cpp}

\subsubsection{Kahn's Algorithm}
\lstinputlisting[style=cppStyle]{Algorithms/Graphs/topological_sort/kahn/main.cpp}

\subsection{Bipartite Graph Check (Undirected Graph)}
$O(n+m)$
\lstinputlisting[style=cppStyle]{Algorithms/Graphs/bipartite_graph_check/main.cpp}

\subsection{Cycle Check (Directed Graph)}
$O(n+m)$
\lstinputlisting[style=cppStyle]{Algorithms/Graphs/cycle_check/main.cpp}

\subsection{Dijkstra}
$O(n \log n + m\log n)$
\lstinputlisting[style=cppStyle]{Algorithms/Graphs/dijkstra/main.cpp}

\section{Math Formulas}
\subsection{Sum of an arithmetic progression}
$S_n = \frac{n}{2}(a_1 + a_n)$

\subsection{Permutation with repeated elements}
$P_n = \frac{n!}{n_1!n_2!...n_k!}$ 

\subsection{Check if is geometric progression}
$a_i^{2} = a_{i-1}a_{i+1}$

\subsection{Bitwise equations}
$a|b = a \oplus b + a\&b
\newline
a\oplus(a\&b)=(a|b)\oplus b
\newline
(a\&b)\oplus(a|b) = a \oplus b
\newline \newline
a+b = a|b + a\&b
\newline
a+b = a\oplus b + 2(a\&b)
\newline \newline
a-b = (a\oplus(a\&b))-((a|b)\oplus a)
\newline
a-b = ((a|b)\oplus b)-((a|b)\oplus a)
\newline
a-b = (a\oplus(a\&b))-(b\oplus(a\&b))
\newline
a-b = ((a|b)\oplus b)-(b\oplus(a\&b))
$

\subsection{Cube of Binomial}
$(a+b)^3 = a^3 + 3a^2b + 3ab^2 + b^3
\newline
(a-b)^3 = a^3 - 3a^2b + 3ab^2-b^3
$

\subsubsection{Sum of Cubes}
$a^3 + b^3 = (a+b)(a^2 - ab + b^2)$

\subsubsection{Difference of Cubes}
$a^3-b^3 = (a-b)(a^2 +ab + b^2)$

\subsection{Binomial expansion}
$\binom{n}{k} = \frac{n!}{k!(n-k)!}
\newline
(a+b)^n = \sum_{k=0}^{n} \binom{n}{k}a^{k}b^{n-k}$

\section{Facts}
\subsection{XOR}
\subsubsection{Self-inverse property}
To cancel a XOR, you can XOR again the same value because $a \oplus a = 0$, so $(value \oplus a) \oplus a = value$

\subsubsection{Identity element}
$a \oplus 0 = a$

\subsubsection{Commutative}
$a \oplus b = b \oplus a$

\subsubsection{Associative}
$(a \oplus b) \oplus c = a \oplus (b \oplus c)$


\end{document}